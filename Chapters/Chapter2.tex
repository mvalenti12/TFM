% Chapter Template

\chapter{Literature Review and Scope of the Thesis} % Main chapter title

\label{Chapter2} % Change X to a consecutive number; for referencing this chapter elsewhere, use \ref{ChapterX}

%----------------------------------------------------------------------------------------
%	SECTION 1
%----------------------------------------------------------------------------------------

\section{Literature Review}

Online crowdfunding and especially Kiva, have received a lot of attention by researchers. In \textcite{Alegre2017}, literature review around crowdfunding is done, focusing on the key management theories around crowdfunding and the principal factors contributing to success for the different crowdfunding models. Following with literature review, in \textcite{Moleskis2018} significant research gaps on crowdfunding are identified. \par
Due to its completeness, data from the Kiva platform has been widely used to do research around it. All the papers followed above include data from this platform in their analyses: In \textcite{Moleskis2016}, the platform is introduced as well as a \textbf{collection of descriptive statistics} about both loans and lenders separately. Following with the use of Kiva's data, \textcite{Canela2017}, the actors on the Kiva Platform are defined, putting a special interest in Africa and the \textbf{role of the partner}. 
\textbf{Loans descriptions} are examined in \textcite{Allison2015}, assessing assess how the linguistic cues in microloan entrepreneurial narratives impact funding outcomes. \par
Exploring the \textbf{lender-borrower relationship}, \textcite{Burtch2013} evaluate the roles of geographic distance and cultural differences have on lenders’ decisions about which borrowers to support. The focus is done on language, difference in GDP, physical distance, cultural differences.
Following with the \textbf{lender-borrower relationship}, \textcite{Galak2010} explores three dimensions of social distance (gender, occupation, and first name initial) between the two agents, concluding that lenders prefer to give to those who are more like themselves. 
In \textcite{Jenq2011}, \textbf{objective loan information, pictures and textual descriptions} are investigated as determinants of individual charitable giving. In it, Research assistants were asked to assess each photograph for qualities such as the borrower’s appearance, age, gender, perceived honesty, and skin color on. \par

Not using Kiva's dataset, \textcite{Pope2011} use data from the website Prosper.com, a leader in online peer-to-peer lending in the United States to end up giving evidence of \textbf{significant racial disparities} by coding variables from pictures and text descriptions. Following with Propsper.com dataset, \textcite{Lin2013} explore on the \textbf{"home bias" topic}, the tendency that transactions are more likely to occur between parties in the same geographical area, rather than outside. \par

Neither using data from Kiva nor Prosper.com but from Kickstarter, \textcite{Greenberg2015} put an emphasis on the figure of the female founder. On the lines of the \textbf{lender-borrower relationship}, they realize that "(female backers) tended to back projects founded by female founders, and this was especially true in technology".

Not in the academic area but worth mentioning, there was recently a \textbf{Kaggle Kernel}\footnote{https://www.kaggle.com/kiva/data-science-for-good-kiva-crowdfunding/version/4} based on the Kiva dataset.
\newpage
\section{Scope of the thesis}
The main objective of the thesis is to \textbf{contribute to the current research in online crowdfunding} by providing insights and analysis on Kiva dataset.
By considering \textcite{Moleskis2018} and for the authors will, the following hypothesis are to be validated:

\underline{On the relationship between lenders and borrowers:}
\begin{tcolorbox}
\textbf{H1:} There are other factors than the ones mentioned in previous research that contribute in the lender decision process.
\end{tcolorbox}

\underline{On descriptions:}
\begin{tcolorbox}
\textbf{H2:} Every partner has a template for their descriptions; being descriptions distinguishable across partners.
\end{tcolorbox}
\begin{tcolorbox}
\textbf{H3:} Partners may change their description template over time and copy other partners.
\end{tcolorbox}

\underline{On borrowers' image:}
\begin{tcolorbox}
\textbf{H4:} Machine Learning can be used to extract expression labels on images.
\end{tcolorbox}

\begin{tcolorbox}
\textbf{H5:} The borrowers face expression on the image has an impact on the loan performance.
\end{tcolorbox}

Last but not least, the main focus and value of this Thesis is the right (both technically and when applicable) application of the different methodologies.