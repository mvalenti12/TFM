% Chapter Template

\chapter{Conclusions} %Main chapter title

\label{Chapter6} 
%Change X to a consecutive number; for referencing this chapter elsewhere, use \ref{ChapterX}

%----------------------------------------------------------------------------------------
%	SECTION 1
%----------------------------------------------------------------------------------------
In this Thesis, we have been focusing on solving completely different problems but with a common topic.\par

In Chapter 2, Literature Review is performed and Opportunities are found. Clear Hypothesis and Goals are stated for this Thesis.\par

In Chapter 3, we focus on the Lender-Borrower relationship. We explore how Female Borrowers receive more funding from Females, and how this differers by Sector. However, due to low interpretability we do not extract valid conclusions. We then provide an approach to describe the categorical relationships on Occupation and Country, rejecting the Null Hypothesis of the $\chi^2$ test of Independence: both Countries and Occupations show strong deviations and therefore associations. When using Correspondence Analysis, it has not been possible to identify \textit{similar} lending behaviour for \textit{similar} countries neither similar lending behaviour for \textit{similar} professions.\par

In Chapter 4, we focus on partners descriptions. We explore how different methods (Metric and Non-Metric Multidimensional Scaling) with different distances (Jaccard, Cosine and Euclidean) are explored to provide a visual representation of text. We finally provide a Framework to evaluate the evolution of descriptions over time. \par
Finally, we showcase how, by using a Suport Vector Machine, partner authorship can be \textit{almost} perfectly attributed. \par
In Chapter 5, we have used Machine Learning to extract Face Expressions from Images. We have later seen how the happiness of a face is properly identified; finding difficulty in identifying other expressions. This is also shown when seeking the importance of Face Expressions in the Loan success: results show that loans whose Face Images have high scores of Happiness fund faster (easily 10\% faster!), but it is not relevant with other emotions. As further actions, considering the interactions with a sentiment output from the descriptions is suggested. \par

Last but not least, another goal has been archieved. This Thesis is public and completely available for reproducibility, containing all the code and files at \url{https://github.com/mvalenti12/TFM}. 